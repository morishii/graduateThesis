\chapter{移動手段判定}
本章では,はじめに人の行動判定に関する関連研究を整理する.
次に,既存研究における問題意識を洗い出す.
最後に,問題意識から必要となる機能要件を導く.
\section{人の行動判定}
本節では,
{\bf 人の動作判定}と
{\bf 人の移動判定}についての研究に分け整理する.
人の行動判定は大きく2つに分類される(表\ref{対象とする行動判定}).
人の動作判定は,人の動作そのものを判定するものであり,
例としては,階段の上り下り,スキップ,ジャンプなどの判定を行う.
2つ目は,人の移動を判定するものである.
例としては,自転車,電車,自動車などの判定を行う.

\begin{table}[htbp]
 \begin{center}
 \begin{tabular}{c|c}
  \hline
  & 対象とする行動判定 \\
  \hline
  人の動作判定 & 歩行,走行,スキップ,ジャンプ,階段の上り下り,座っている状態,立っている状態など	 \\
  人の移動判定 & 歩行,走行,自転車,電車,バス,自動車など \\
  \hline
 \end{tabular}
 \end{center}
 \caption{対象とする行動判定}
 \label{対象とする行動判定}
\end{table}

\subsection{人の動作判定}
人の動作判定の研究例としてはKwapisz, Weiss, Mooreらの研究\cite{Kwapisz:2011:ARU:1964897.1964918}があげられる.
この研究では,脚につけた加速度センサを用いて,歩行,ジョギング,階段上り,階段下り,座っている状態,立っている状態の6種類の行動を分類している.
J48, Logistic Regression, Multilayer Perceptronの3種類の機械学習アルゴリズムを用いて動作の判定を行い,
Multilayer Perceptronを用いた場合に91.7\%の精度で分類している.
また,倉沢,河原,森川,青山らの研究\cite{倉沢央:2006-05-23}では,
3軸加速度センサのデータから分散値,平均値,FFTを用いることにより端末の装着位置を推定し,
さらに座る,立つ,歩行,走行の動作を判定している.
品川,谷川,太田らの研究\cite{weko_12507_1}では,
独居高齢者の在宅行動モニタリングを行うため,
3軸加速度センサを用いて各軸のパワースペクトルを算出し,歩行や転倒を検出している.
北沢らの研究\cite{北沢俊二200210}では,
1軸加速度センサと1軸ジャイロセンサを搭載した腕時計型デバイスを作成し,
歩行,走行,ジョギング,軽いスポーツや,激しいスポーツなどの状態を検出している.


\subsection{人の移動判定}
人の移動判定の研究例としては小林,岩本,西山らの研究\cite{釈迦}があげられる.
この研究では,加速度センサ,マイク,GPSを用いて歩行,走行,自転車,停止,自動車,バス,電車の7種類の移動手段を対象とし,ユーザの移動状態を高精度で分類することが可能である.
しかし,GPSを用いた判定では,GPS電波の届かない地下鉄などでは判定できない問題や,
複数センサの併用による消費電力の増大が問題と考えられる.
スマートフォンのみを用いて日々の移動手段を記録するためには,これらの課題克服が必要とされている.
また,山崎,五味田らの研究\cite{山崎亜希子:2008-03-13}では,
カバンの中に設置した加速度センサを用いて移動手段を分類している.
しかし,センサの装着位置をカバンの中という限定された場所でセンサ情報を取得しているために,
ユーザの日常的なシステムの利用は困難である.
そして,池谷,菊池,長,服部らの研究\cite{池谷直紀:2008-07-10}では,
携帯端末の3軸加速度データからニューラルネットワークを用いて
running(走行), walking(歩行), still(静止), vehicle(電車およびバスの移動)の移動手段を判定し,約87\%の精度で分類している.

本研究では,2つ目の人の移動の判定を研究対象とする.

\section{問題意識}
前節において,人の行動判定についての関連研究を人の動作の判定の研究と人の移動の判定の研究に分け整理した.
本節では,関連研究における問題意識を整理する.
\subsubsection*{GPSを用いているため地下などでは判定できない問題}
小林,岩本,西山らの研究では移動手段の判定のためのセンサの1つとしてGPSを用いている.
しかし,GPSを用いてしまうとGPSを使えない場所が存在する.
例えば,地下鉄や屋内などGPS電波が届かない場所ではGPSを使用することが出来ない.
そのため移動手段の判定する際にGPSに依存してしまうとGPSを使用することが出来ない場合に,移動手段を判定することが不可能になる.
よって,あらゆる場所で移動手段を判定するシチュエーションを想定した場合には
GPSを用いることは適切ではない.
\subsubsection*{複数センサの使用によるスマートフォンの消費電力の増大}
また,小林,岩本,西山らの研究では移動手段の判定のために加速度センサ,マイク,GPSなどのセンサを用いている.
GPSを使用する際には測位のために計算をする必要があり,スマートフォンの消費電力の増大が考えられる.
また複数のセンサを複合的に用いると,使用するセンサが増えるごとにスマートフォンの消費電力が増えると考えられる.
デバイスの消費電力が増えてしまうと,長時間移動手段を判定することが出来なくなり,ユーザが日常的に移動手段を記録することが困難になる.
\subsubsection*{スマートフォンの所持位置が限定されている問題}
山崎,五味田らの研究では加速度センサの位置をカバンの中という限定された場所でのみ移動手段の判定を行っている.
しかし,実際のユーザが使用している場合を考えた場合には,カバンの中だけではなく,ズボンのポケットの中や,胸ポケット,あるいは手に持っている場合など様々な所持位置がある.
あらゆるユーザが移動手段を判定する際に,所持位置がカバンの中だけに限定されていることは現実的なシステムではない.
多様なユーザに適用するためには,様々なセンサの所持位置に対応する必要がある.

\section{機能要件}
本節では,前節で述べた問題意識を踏まえ,移動手段判定システムにおける
既存研究の問題点を解決するための機能要件を述べる.
本論文における要件としてあらゆる場所において低消費電力で移動手段を判定すること,
多様なユーザに適用することができることの2つを提案する.

\subsubsection*{あらゆる場所において低消費電力で移動手段を判定}
本システムでは,ユーザがあらゆる場所において移動手段を判定できること,
また低消費電力で移動手段を判定し続ける事ができ,ユーザが日常的に長時間移動手段を判定できる必要がある.
既存研究では,複数のセンサを用いて移動手段を判定している.
複数のセンサを用いることにより高精度で移動手段を判定することができる.
しかし,高精度な移動手段の判定と引き換えに,あらゆる場所で判定を行うことが出来なかったり,消費電力が増えてしまったりしている.
これらの問題点のため,日常生活で移動手段を判定するシステムを使用することは難しい.
本システムにより,日常生活で実用的な移動手段を判定するシステムを提供することが可能になる.

\subsubsection*{様々なスマートフォンの所持位置で判定}
移動手段の判定をする際には,スマートフォンの様々な所持位置に対応する必要がある.
利用者により,スマートフォンの所持位置は異なるため,
既存研究のように所持位置をカバンの中と1ヶ所に限定してしまうと,
限られた装着位置でしか移動手段を判定することができず,多様なユーザに適用することができない.
なるべく多くの所持位置で移動手段を判定できる必要がある.

\section{まとめ}
本章では,はじめに人の行動判定に関する関連研究を人の動作の判定と人の移動の判定に分け整理した.
そして,既存の移動手段の判定における問題意識を洗い出し,移動手段判定システムの機能要件を導き出した.
次章では,問題意識に基づく要件に対するアプローチを示し,本システムの目的と特徴を述べる.