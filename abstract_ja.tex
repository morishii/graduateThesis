\begin{center}
\textbf{\Large 卒業論文要旨 2016年度(平成28年度)}

\vspace{6.18mm}

\textbf{\Large 遠隔UAV操作における複数人称視点を実現する飛行モデルの提案}
\end{center}

\vspace{10mm}

\begin{flushleft}
\textbf{論文要旨}\\
\end{flushleft}

近年のセンサやハードウェアデバイスの発展により、空中を自由に移動可能なUAV機構の普及が進んでいる。
それを踏まえて,自動飛行、センシング手法とともに、遠隔で現実空間に実際に干渉する遠隔作業を行うドローンの研究が進んでいる.操作性については,言及されていることが少ない.その中でも,ドローンの遠隔作業において,操作を支援するモデルを提案する.アプローチとしては、視点を拡張することに着目し,複数台のドローンを協調して動作させるモデルを提案する.

% 近年のセンサの小型化・高性能化により,スマートフォンやウェアラブルデバイスなど
% のデバイスに様々なセンサが搭載されている.
% スマートフォンやウェアラブルデバイスが普及し,個人が日常的に様々なセンサを身につけ,利用できるようになった.
% それに伴って,日々の行動記録など日常的な行動を記録する研究が盛んになっている.
% 移動手段の判定もその研究の一部である.
% 移動手段の判定ができるようになると,日々の移動に関する行動を記録できるだけではなく,
% 移動手段に合わせた様々なアプリケーション提供が可能になる.
% 例えば,電車やバスなどの公共交通機関で移動している場合にはスマートフォンを
% 自動的にマナーモードに設定するアプリケーションや
% ランニング中に自動で音楽を流すアプリケーションなど
% が考えられる.
% 既存の研究では,加速度センサやマイク,GPSなど複数のセンサを用いて高精度で移動手段を判定している.
% しかし,GPSを用いているため地下など電波が届かない場所では判定できないという問題や,
% 複数のセンサを使用しているためデバイスの電池の消耗が激しいという問題があり,
% さらに,所持場所がカバンの中など,1つの場所に限定されてしまうという問題もあるため,スマートフォンのみを用いて日常的に移動手段を判定することは実用的ではない.


% 本研究では,あらゆる場所において低消費電力で移動手段を判定し,所持位置に限定せずに移動手段を判定する,
% 加速度センサのみを用いた移動手段判定システムを提案する.
% 移動手段判定システムは,
% 加速度データ取得モジュール,移動手段判定モジュール,移動手段補正モジュールの3つのモジュールから構成され,
% 加速度センサのみを用いてスマートフォン上でリアルタイムに移動手段の判定を行う.
% 加速度データ取得モジュールでは,スマートフォンの加速度センサから加速度データを取得する.
% 移動手段判定モジュールでは,加速度データから時系列分割手法や周波数分割手法を用いて特徴量の抽出を行い,機械学習を用いて移動手段の分類を行う.
% 移動手段補正モジュールでは,移動手段判定モジュールで判定された移動手段を前後の判定された移動手段を比較する事によって補正を行う.
%機械学習を用いるとか特徴量に何を使うかとかを書く

% 評価実験として,提案した移動手段判定システムに対して
% 被験者5名を対象として実環境における本システムの移動手段の分類精度を測定し,本システムの有用性を評価する.
% 評価実験から,本システムを使用し移動手段を98.44\%で判定することが出来た.
%何人を対象として結果とか具体的に
%今後の展望も書く

\begin{flushleft}
\textbf{キーワード}\\
\textbf{遠隔操作、UAV、協調制御、視点拡張}

\end{flushleft}

\begin{flushright}
\textbf{慶應義塾大学 環境情報学部}\\
\textbf{森重 浩直}
\end{flushright}
\newpage

