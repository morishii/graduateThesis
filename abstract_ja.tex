\begin{center}
\textbf{\Large 卒業論文要旨 2016年度(平成28年度)}

\vspace{6.18mm}

\textbf{\Large 遠隔UAV操作における複数人称視点を実現する飛行モデルの提案}
\end{center}

\vspace{10mm}

\begin{flushleft}
\textbf{論文要旨}\\
\end{flushleft}



% 背景      ドローンの普及
% 問題      遠隔操作によるフィードバックループの欠如
% 着眼点     視点を拡張する
% 解決アイディア 視点拡張を実現する複数台協調の飛行モデルの提案
% 解決手法    画像マーカーをつけて俯瞰、鳥瞰を移動できるアプリケーションの実装
% 予備実験    複数人称の視点を提供することで操作性が向上するかの実験
% 評価      追尾モジュール、視点移動モジュールについて評価する



近年,ハードウェアデバイスの小型化・高性能化により,PepperやRoombaなどの人間を物理的に支援するロボットが普及している.その中でも,空中を自由に移動可能な無線操縦飛行機(ドローン)の発展が進んでおり,農薬散布や荷物運搬などの実用的なものから,昨今ではアートなどのエンターテインメントの領域にまで活躍の場を広げている.その著しい発展の中の一つの潮流として,ドローンが現実空間に干渉し,作業を支援する「マニピュレーション型」のドローンの研究が盛んになっている.既存の研究では建設補助を行なったり,ロボットアームを取り付けて文字を描く研究などの具体的なアプローチが行われているが,これらは問題として,人間の目の届かない場所でドローン飛ばす「目視外飛行」を想定していない点が挙げられる.そのために,目の届かない遠隔地での操作が要求される,極地環境における生態調査や放射線汚染地域での浄化作業などのシナリオにおいては未だに十分な操作性が見込めず,利用用途が限定されてしまっている.本研究では,そういった遠隔地での操作におけるユーザの操作性を支援するための飛行モデルを提案する.アプローチとして,情報量の最も多い視覚情報に着目し,「視点拡張」を行う複数台による協調制御モデルを考案した.また,上記のモデルを適用したプロトタイプとして任意の点に視点を移動することが可能なドローンカメラワーク編集システム「Arbi-EYE」を製作した.


本システムは,作業を行う「親ドローン」と,それに追尾し「上から・横から・後ろから」の視点を提供し続ける「子ドローン」の二台から構成されている.本システムは,追尾モジュール,視点移動モジュールの二つから構成されており,非外部依存な構造を持ちながら遠隔のユーザに多様な視点を提供することが可能だ.視点移動モジュールは,ユーザーの指定した視点に対して親と子の相対的な位置関係を保ちながら移動する指令を送り,追尾モジュールにおいては,親ドローンに対して常に一定の間隔で追尾する信号を送る.


本研究では,まず予備実験としてドローンにおける最適な視点拡張モデルについて考察し,それを実現する二つのモジュールについての性能評価を行い,最後に完成したプロトタイプを○名のユーザに使用してもらうユーザースタディを実施する.
結果として,既存の単一のドローンによる遠隔操作よりも(作業効率が向上)したことで,本システムの有用性を示すことに成功した.





% 近年のセンサの小型化・高性能化により,スマートフォンやウェアラブルデバイスなど
% のデバイスに様々なセンサが搭載されている.
% スマートフォンやウェアラブルデバイスが普及し,個人が日常的に様々なセンサを身につけ,利用できるようになった.
% それに伴って,日々の行動記録など日常的な行動を記録する研究が盛んになっている.
% 移動手段の判定もその研究の一部である.
% 移動手段の判定ができるようになると,日々の移動に関する行動を記録できるだけではなく,
% 移動手段に合わせた様々なアプリケーション提供が可能になる.
% 例えば,電車やバスなどの公共交通機関で移動している場合にはスマートフォンを
% 自動的にマナーモードに設定するアプリケーションや
% ランニング中に自動で音楽を流すアプリケーションなど
% が考えられる.
% 既存の研究では,加速度センサやマイク,GPSなど複数のセンサを用いて高精度で移動手段を判定している.
% しかし,GPSを用いているため地下など電波が届かない場所では判定できないという問題や,
% 複数のセンサを使用しているためデバイスの電池の消耗が激しいという問題があり,
% さらに,所持場所がカバンの中など,1つの場所に限定されてしまうという問題もあるため,スマートフォンのみを用いて日常的に移動手段を判定することは実用的ではない.


% 本研究では,あらゆる場所において低消費電力で移動手段を判定し,所持位置に限定せずに移動手段を判定する,
% 加速度センサのみを用いた移動手段判定システムを提案する.
% 移動手段判定システムは,
% 加速度データ取得モジュール,移動手段判定モジュール,移動手段補正モジュールの3つのモジュールから構成され,
% 加速度センサのみを用いてスマートフォン上でリアルタイムに移動手段の判定を行う.
% 加速度データ取得モジュールでは,スマートフォンの加速度センサから加速度データを取得する.
% 移動手段判定モジュールでは,加速度データから時系列分割手法や周波数分割手法を用いて特徴量の抽出を行い,機械学習を用いて移動手段の分類を行う.
% 移動手段補正モジュールでは,移動手段判定モジュールで判定された移動手段を前後の判定された移動手段を比較する事によって補正を行う.
%機械学習を用いるとか特徴量に何を使うかとかを書く

% 評価実験として,提案した移動手段判定システムに対して
% 被験者5名を対象として実環境における本システムの移動手段の分類精度を測定し,本システムの有用性を評価する.
% 評価実験から,本システムを使用し移動手段を98.44\%で判定することが出来た.
%何人を対象として結果とか具体的に
%今後の展望も書く

\begin{flushleft}
\textbf{キーワード}\\
\textbf{遠隔操作、UAV、協調制御、視点拡張}

\end{flushleft}

\begin{flushright}
\textbf{慶應義塾大学 環境情報学部}\\
\textbf{森重 浩直}
\end{flushright}
\newpage

