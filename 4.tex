\chapter{設計}
本章では,まず提案する視点拡張システムの設計概要について説明する.
ついで,ソフトウェアのモジュールの設計について説明する.

% \section{本システムの設計概要}
% 本研究では,ユーザが日常的に移動手段を判定し記録し続けるために,
% スマートフォン上で動作する移動手段判定システムを提案する.
% \begin{figure}[htbp]
%   \begin{center}
%    \includegraphics[width=150mm]{image/4/system.eps}
%    \caption{システム構成図}
%    \label{system}
%   \end{center}
% \end{figure}
% 本システムのシステム構成図を\ref{system}に示す.
% 本システムは加速度データ取得モジュール,移動手段判定モジュール,移動手段補正モジュールの3つのモジュールから構成されている.
% 加速度データ取得モジュールでは,スマートフォンの加速度センサから加速度データの取得を行う.
% 移動手段判定モジュールでは,加速度データから時系列分割手法や周波数分割手法を用いて特徴量を抽出し,教師あり機械学習を用いて移動手段の判定を行う.
% 移動手段補正モジュールでは,移動手段判定モジュールで判定された移動手段を前後の判定された移動手段を比較する事によって補正を行う.

% \section{移動手段判定モジュール}
% 移動手段判定モジュールでは,教師あり機械学習を用いて移動手段の判定を行う.
% 具体的には,加速度センサ取得モジュールから3軸加速度データを取得し,時系列分割手法や周波数分割手法を用いて特徴量を抽出し,機械学習によって移動手段の判定を行う.

% \begin{figure}[htbp]
%   \begin{center}
%    \includegraphics[width=150mm]{image/4/system2.eps}
%    \caption{移動手段判定モジュール}
%    \label{fig2}
%   \end{center}
% \end{figure}
% 図\ref{fig2}に移動手段判定モジュールにおける2つの動作フェーズ,
% 学習フェーズ(点線),分類フェーズ(実線)を示す.
% 学習フェーズにおいては,機械学習に適用する加速度データを取得し,
% 特徴量を抽出した後,その教師データをもとに分類器のモデルを作成する.
% 一方,分類フェーズでは,
% 取得した加速度データから特徴量を抽出し,
% 学習フェーズで作成したモデルを用いて,実際に移動手段をスマートフォン上で逐次判定する.

% \begin{table}[t]
% \caption{特徴量の抽出手法}
% \centering
% \label{featurelist}
% \begin{tabular}{|l|c|}
% \hline
%  & 平均 ($\bar{x}, \bar{y}, \bar{z}$)\\
% 時系列分割 & 二乗平均平方根 ($\sqrt{\bar{x}^2+\bar{y}^2+\bar{z}^2}$)\\
%        & 分散 \{$var(x), var(y), var(z)$\}\\
% & 相関係数 \{$corr(x,y), corr(y,z), corr(x,z)$\}\\
% & 共分散 \{$cov(x,y), cov(y,z), cov(x,z)$\} \\ \hline
% 周波数分割 & FFT power spectrum ($p_1, p_2, \cdots, p_n$) \\
% \hline
% \end{tabular}
% \end{table}

% 特徴量の抽出手法としては,表\ref{featurelist}にあげる,加速度センサを用いた行動認識研究において広く使われる特徴量\cite{6246136}を用いる.
% それらは時系列分割及び周波数分割の2種類から構成される.
% 時系列分割としては,3軸加速度データの平均,二乗平均平方根,分散,相関係数,共分散を用いる.
% 周波数分割としてはフーリエ変換を用いる.

% また,教師あり機械学習を用いて移動手段の判定を行う.
% あらかじめ取得した加速度データから特徴量を抽出したものを教師データとしてセットする.
% その後,機械学習によってリアルタイムに移動手段を判定する.
% 本研究では,機械学習アルゴリズムとしてBayesNet\cite{BayesNet}, J48\cite{J48}, KStar\cite{KStar}, Multilayer Perceptron, Random Forest\cite{RandomForest}, SMO\cite{SMO}の6種類のアルゴリズムを使用する.

% \section{移動手段補正モジュール}
% 移動手段補正モジュールは,移動手段判定モジュールで判定された移動手段が本当に信頼できるかを確認し,補正する.
% 移動手段補正モジュールでの補正例を図\ref{filter1}に示す.
% 具体的には前後の判定された移動手段を比較し,前後の移動手段が同じで,今回判定された移動手段が異なった場合は,今回判定された移動手段を補正する.
% 例えば,1秒間に1回移動手段を判定し,3秒間の間に「歩行→電車→歩行」と判定された場合,
% ごく短い時間で歩行から電車に変わることは考えにくい.
% そのためこの判定内容を「歩行→歩行→歩行」と補正し,最終的な移動手段と決定する.
% \begin{figure}[htbp]
%   \begin{center}
%    \includegraphics[width=150mm]{image/4/filter.eps}
%    \caption{移動手段補正モジュール}
%    \label{filter1}
%   \end{center}
% \end{figure}


% \section{スマートフォンの所持位置}
% 既存研究での移動手段を判定する際にカバンの中など,センサの所持位置が限定されているという問題があった.
% 本研究では多様なユーザに適用するため,本システムでは複数の所持位置に対応する.
% 具体的にはズボンのポケットの中,カバンの中,手の上の3つの所持位置で本システムを利用できるようにする.
% \begin{figure}[htbp]
%   \begin{center}
%    \includegraphics[width=100mm]{image/4/photo01.png}
%    \caption{スマートフォンの所持位置【ズボンのポケットの中】}
%    \label{photo1}
%   \end{center}
% \end{figure}
% \begin{figure}[htbp]
%   \begin{center}
%    \includegraphics[width=100mm]{image/4/photo02.png}
%    \caption{スマートフォンの所持位置【カバンの中】}
%    \label{photo2}
%   \end{center}
% \end{figure}
% \begin{figure}[htbp]
%   \begin{center}
%    \includegraphics[width=100mm]{image/4/photo03.png}
%    \caption{スマートフォンの所持位置【手の上】}
%    \label{photo3}
%   \end{center}
% \end{figure}

% イメージ図をそれぞれ図\ref{photo1},図\ref{photo2},図\ref{photo3}に示す.
% 複数の所持位置で移動手段の判定を行うために,本システムでは予めそれぞれの所持位置での加速度データを収集し,特徴量を抽出し教師データを作成する.
% 収集する移動手段ごとの所持場所の加速度データを表\ref{positionlist}にまとめる.
% なお,本研究では安全のため自転車と車の場合,手の上で移動手段を判定することを想定していない.
% \begin{table}[htbp]
%  \begin{center}
%  \caption{収集する加速度データ}
%  \label{positionlist}
%   \begin{tabular}{c|c}
%   \hline
%    & ズボンのポケットの中 \\ \cline{2-2}
%   歩行 & カバンの中 \\ \cline{2-2}
%    & 手の上 \\ \hline
%    & ズボンのポケットの中 \\ \cline{2-2}
%    走行 & カバンの中 \\ \cline{2-2}
%    & 手の上 \\ \hline
%   自転車 & ズボンのポケットの中 \\ \cline{2-2}
%    & カバンの中 \\ \hline
%    & ズボンのポケットの中 \\ \cline{2-2}
%   電車 & カバンの中 \\ \cline{2-2}
%    & 手の上 \\ \hline
%    & ズボンのポケットの中 \\ \cline{2-2}
%   バス & カバンの中 \\ \cline{2-2}
%    & 手の上 \\ \hline
%   自動車 & ズボンのポケットの中 \\ \cline{2-2}
%   & カバンの中 \\ \hline
%    & ズボンのポケットの中 \\ \cline{2-2}
%   その他 & カバンの中 \\ \cline{2-2}
%    & 手の上 \\
%   \hline
%   \end{tabular}
%  \end{center}
% \end{table}


% \section{まとめ}
% 本章では,移動手段判定システムの設計について述べた.
% 次章では,加速度センサのみを用いた本システムの実装について説明する.