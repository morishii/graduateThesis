\chapter{序論}
本章では,はじめに本研究における背景を述べる.
ついで,本研究の目的を述べる.
最後に本論文の構成を示す.

\section{背景}
\subsection{ドローンの発展と有用性}
近年,センサーやアクチュエータの小型化・高性能化により,人間の活動を物理的に支援し拡張するロボットが増えてきている.中でも今一大産業として注目されているのが,いわゆるドローンと呼ばれる無人航空機(UAV)だ.矢野経済研究所の調査によると(図\ref{dronePotential}),ドローンによる世界市場規模は軍事用のものを含めると2015年の時点で1兆2410億円,2020年には2兆2814億円にまで成長すると予測している\cite{Yano}.民間のドローンにおいても現在は4053億円だが,2020年までには比率が軍事用とほぼ同じになると予測している.

\begin{figure}[htbp]
 \begin{center}
  \includegraphics[width=100mm]{image/1/dronePotential.eps}
 \end{center}
 \caption{ドローンの世界市場の成長予測}
 \label{dronePotential}
\end{figure} 

「空の産業革命」としてしばしば取りざたされているドローンの特徴として,地形にとらわれず自由度の高い三次元の動きを可能とし,人間の到達できない領域にアクセスできる点が挙げられる.機構による制限はあるが,被災地や洞窟などの極地環境や,成層圏までの高さまで上昇することが可能だ.
それによって現在では,マスメディアにおいては速報を空撮するために利用し,農業においては赤外線センサーを搭載して害虫の早期発見などを行うことで農作物の収量と品質を管理したりといったように,様々なアプリケーションがすでに実現段階にきている.農業,医療,在庫管理,運搬,スポーツ,エンターテインメントなど,その活用は昨今では多岐にわたる(図\ref{application}).

人間を直接的に支援するPepperやRoombaなどのソーシャブルなロボットとしてもドローンは利用されるポテンシャルを持っており,パーソナルドローンとしての開発も進んでいる\cite{Fleye}.将来ドローンはスマートフォンにはできなかった情報提示以上の物理的な恩恵を与えるデバイスとして普及することが想定される.

\begin{figure}[htbp]
 \begin{center}
  \includegraphics[width=100mm]{image/1/manipulation.eps}
 \end{center}
 \caption{ドローンのアプリケーション一覧 }
 \label{application}
\end{figure} 

このように,一つのプラットフォームとして受け入れられつつあるドローンにおける一つの潮流としてセンシングを行うだけでなく,現実空間に干渉する「マニピュレーション型」のドローンが盛んに研究されている.
Augugliaroらによる研究では\cite{augugliaro2014flight}\cite{augugliaro2013flight},建築資材であるロープやブロックを複数台により協調して組み上げて,人の手を使わずに建築物を建設している.また,AIRobotsプロジェクトでは\cite{marconi2012aerial},ドローンにロボットハンドをつけて運搬,ペンによる図形の描画などのマニピュレーションを行う試みが実施されている.これらの研究がマニピューレーション型のドローンの普及を加速されている.

\section{問題意識}
前述したマニピュレーション型のドローンの研究が進む中で,一つの問題として遠隔作業の分野における拡張性の低さが挙げられる.ロボットハンドによる何らかの操作や,農薬の散布などのマニピュレーションは,厳密な操作が必要とされる.仮に操縦者が近くにいることができれば,作業の場面を目視して操作することができるが,一定のシナリオにおいては(極地環境における生態調査などの,ドローンの柔軟な機動性を活かした屋外や遠隔操作におけるシナリオ)ドローンに付帯させるカメラのみを頼りに操作する「目視外飛行」をするケースは少なくない.ドローンはその機動性や動きの柔軟性を優先するがために,センサを多量に積むことができず,ドローン単体では得ることのできる環境情報が少ない(図\ref{problem}).つまり、作業の操作性が落ちてしまう.望遠レンズなどで視界を広げることは一つのアプローチとして考えられるが,本研究では,最も情報量が多いとされる視覚の中でも,視点が一つしかないことに着目した.何かを操作するにあたって作業者自身,作業対象,周囲の様子をリアルタイムで知る必要があり,それは一つのドローンでは難しい.

\begin{figure}[htbp]
 \begin{center}
  \includegraphics[width=100mm]{image/1/problem.eps}
 \end{center}
 \caption{目視外飛行のイメージ}
 \label{problem}
\end{figure} 

% ・ドローンの産業の拡大における発展
% ・政治的な背景(軍事的な背景からの誕生が,「空の産業革命」へ)
% ・多くのセンシングができるようになってきたり,オートパイロットのアルゴリズムなどの研究が行われてきており,実用段階まできているものもあるが
% ・その中でもマニピュレーションに関する研究が盛んになってきている.
% ・ethやAIRobotsなどの一連のプロジェクトは,現実空間に干渉するために
% ・しかし,その中で外部環境に依存できないため,現在は遠隔で何か作業をする際は非常に限られたミッションしかできていない
% ・問題として,ドローンの単体による飛行や,センサーの数とその稼働範囲が狭いことがあげられる
% ・本研究では操作性を向上させるアプローチとして「視点」を拡張させることに着目した

% 今日では様々なものがインターネットによって繋がり,デジタル技術が人間の活動を支援する世界が到来している.
% 安倍政権の提示したSociety5.0でも超スマート社会を謳っているように,日本のみならず世界全土でテクノロジーを駆使した社会を実現しようとしている.


% 近年,FuelBand\cite{FuelBand}やFitbit\cite{Fitbit}, Android Wear\cite{AndroidWear}などのウェアラブルデバイスやスマートフォンの普及により,
% 加速度センサや角速度センサ,GPSなど様々なセンサを利用できるようになった.
% \begin{figure}[htbp]
%  \begin{center}
%   \includegraphics[width=100mm]{image/1/smartphone.eps}
%  \end{center}
%  \caption{世界のスマートフォンユーザ数の推移(推計値)\cite{Smartphone}}
%  \label{smartphone}
% \end{figure}
% 総務省の平成26年度版 情報通信白書\cite{Smartphone}によると(図\ref{smartphone}),スマートフォンの普及が進んでおり,
% 2013年におけるスマートフォンの世界出荷台数は前年比の38.4\%増え,年間出荷台数が初めて10億台を超えた.
% また,出荷台数の増加とともにスマートフォンのユーザ数も増加している.
% スマートフォンのユーザ数は2012年に11.3億人だったものが,2014年には17.5億人になっている.
% また,2017年には25億人になると推計されている.
% 携帯電話ユーザに占める割合も2012年に27.6\%だったものが,2014年には38.5\%になっている.
% 携帯電話を使用しているユーザがフィーチャーフォンからスマートフォンへ移り変わっているため,
% 今後もスマートフォンのユーザ数は増え続けるだろうと予測される.

% それに伴い,日常的な行動を記録するライフログに関する研究が盛んになっている.
% 食事の内容や活動量の記録,移動手段など様々なライフログの取得について研究がされている.
% FoodLog\cite{FoodLog}は,毎日の食事の内容を記録できるサービスである.
% これは,毎日の食事の写真を撮り送信することで,毎日の食事の内容を記録し,
% 写真から栄養バランスを自動で推定し,健康管理も出来るというサービスである.
% また,FuelBandは毎日の活動量を記録できるデバイスである.
% これは,ランニングや日常生活のあらゆる動きをFuelという単位に換算し,1日の活動量を記録する.
% 様々なライフログに注目したサービスが生まれている.

% 本研究では,移動手段の判定を対象とする.
% 移動手段を判定することで,
% 歩行や走行などの判定によって,日々の運動量を測定することや
% ユーザの各コンテキストに合わせたアプリケーション提供が可能になる.
% 例えば,電車やバスなどの公共交通機関で移動している場合にはスマートフォンを
% 自動的にマナーモードに設定するアプリケーションや,
% ランニングの時に音楽を自動で再生するアプリケーションなどが考えられる.

% 日常行動記録として実際に移動手段を記録するシステム例として「Moves」\cite{Moves}がある.
% 「Moves」はスマートフォン上で動作し,1日の移動をWalk(歩行), Run(走行), Cycling(自転車), Transport(交通機関)の4種類に分けて記録をし,
% 歩数を測ったり,走った距離を記録したりできるアプリケーションである.
% しかし,交通機関という大きな括りでしか分類をすることが出来ない.
% 各移動手段に合わせたアプリケーションを提供するためには,
% 交通機関を電車やバスのように細かく分類する必要がある.

% \begin{figure}[htbp]
%  \begin{center}
%   \includegraphics[width=50mm]{image/1/moves.eps}
%  \end{center}
%  \caption{Moves\cite{Moves}}
%  \label{moves}
% \end{figure}


\section{目的}
本研究では,ドローンにおける研究の流れを整理し,遠隔操作における課題を洗い出した上で,遠隔操作支援を行うための最適な視点拡張モデルを作成することを目的とする.
% 具体的には,二台のドローンを協調させて片方のドローンに俯瞰視点や上からみた視点などを状況に合わせて提供するカメラワーク編集システムを作成する.

% 本研究では,スマートフォンを用いて日常的に移動手段を判定するシステムを提案する.
% 既存の研究では,加速度センサやマイク,GPSなど複数のセンサを用いて移動手段を判定しているが,
% スマートフォン上の複数のセンサを用いて移動手段を判定する場合,
% スマートフォンの消費電力が大きいため,ユーザが日常的に移動手段を判定することは現実的ではない.
% そのため,本研究では加速度センサのみを用いて,歩行,走行,自転車,電車,バス,自動車,その他(静止している状態を含む)の7種類の移動手段を分類する.
% 加速度センサのみを用いることにより,スマートフォンの省電力性を向上させ
% ,GPSを用いないため,地下鉄など場所に左右されないで長時間にわたって移動手段の判定が可能になる.



\section{構成}
本論文は,本章を含め全8章からなる.
本章では,本研究における背景と問題意識,目的を述べ,貢献したことを明確にした.
第2章では,本研究における遠隔操作支援について整理し,機能要件を導出する.
第3章では,本研究の機能要件を満たす視点拡張モデルを提案し,本システムの目的と特徴を述べる.
第4章では,本システムの設計について述べる.
第5章では,本システムの実装について述べる.
第6章では,本システムで用いる予備実験について述べる.
第7章では,本システムの精度を評価し,考察を述べる.
第8章では,本論文の結論と今後の展望について述べる.

\section{まとめ}
本研究における貢献は以下の3つである.
\begin{itemize}
 \item ドローンの遠隔操作における最適な視点についてのスタディ
 \item 視点を拡張するシステムの実装
 \item 遠隔操作における操作性を向上させる飛行モデルの提案
\end{itemize}


% 本論文は,本章を含め全8章からなる.
% 本章では,本研究における背景と問題意識,目的を述べた.
% 第2章では,本研究における移動手段の判定について整理し,問題意識を洗い出し,機能要件を導く.
% 第3章では,本研究の機能要件を満たす移動手段判定システムを提案し,本システムの目的と特徴を述べる.
% 第4章では,本システムでのデバイスの装着位置や機械学習の設計について述べる.
% 第5章では,本システムの実装について述べる.
% 第6章では,本システムで用いる特徴量と機械学習アルゴリズムの選定のため行った予備実験について述べる.
% 第7章では,本システムの精度を評価し,考察を述べる.
% 第8章では,本論文の結論と今後の展望について述べる.
