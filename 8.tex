\chapter{結論}
本章では,本研究における今後の展望と本論文のまとめを述べる.

\section{今後の展望}
% 本節では,本研究が提案した加速度センサのみを用いた移動手段判定システムの設計・実装における問題点と解決策を述べ,今後の展望を述べる.

% \subsection{多様な移動手段の判定}
% 本研究では,提案した移動手段判定システムで判定する移動手段は,
% 小林,岩本,西山らの先行研究を参考にし,
% 「歩行,走行,自転車,電車,バス,自動車,その他(移動していない状態を含む)」の7種類とした.
% しかし,実際の生活の中では,これ以外にも様々な移動手段がある.
% 例えば,新幹線や飛行機,船などの移動手段が考えられる.
% ユーザのあらゆる移動手段に対応することで,より広い応用ができると考えられる.

% 問題点としては,判定する移動手段が増えることで,
% 移動手段判定モジュールでの機械学習で分類する種類が増えてしまうため,分類精度が低下してしまうと考えられる.解決策として,2つの方法が考えられる.
% 1つ目は,あらゆる移動手段の教師データの作成である.
% 教師データを作成しなければ分類することが出来ないため,
% まずは様々な移動手段の加速度データを収集し,教師データを作成する必要がある.
% 2つ目は,新たな特徴量の抽出手法を考えることである.
% 判定する移動手段の種類を増やすと,移動手段の分類精度が低下してしまうため,
% 現在の時系列分割手法に加える新たな特徴量を考える必要がある.

% \subsection{ユーザごとの教師データの作成}
% 本論文で行った評価実験の結果から,
% 教師データがないユーザを追加した場合の分類精度は9.78\%と
% 分類精度が大きく低下しまうという問題点があった.
% 本研究が提案するシステムを広く使用する際には,
% その人ごとに教師データを作成しなければいけないという設計では実用的ではない.
% 教師データを作成しなくてもある程度の精度で移動手段を判定できる必要がある.

% 解決策としては,本論文で行った評価実験では,
% 教師データを生成した人数が少なすぎたことが1つの原因であると考えられるため,
% 教師データとする加速度データを多くの人数で収集する必要があると考えられる.
% 人の動きはある程度のパターンがあると考えられるので,
% まずは20人程度の加速度データから教師データを作成し,同じように分類精度を比較しようと考えている.
% また,本システムでは,あらかじめシステム内に教師データが登録されているという設計のため,
% 新たに使用するユーザが後からこの時間にこの移動手段を行っていたという様に登録出来る機能を追加し,簡単に教師データを追加できるような仕組みを作成する.

\section{本論文のまとめ}
% 本論文では,スマートフォンの加速度センサのみを用いて移動手段を判定するシステムの提案を行った.
% 加速度センサから取得した加速度データから時系列分割手法や周波数分割手法により特徴量を抽出し,
% 7種類の機械学習アルゴリズムから本システムに最適な特徴量と機械学習アルゴリズムを選定した.
% 本システムでは特徴量の抽出に時系列分割手法を用い,機械学習アルゴリズムにRandom Forestを用いて移動手段の判定を行う.
% 評価実験から,教師データがあるユーザの場合に
% 移動手段補正モジュールを用いずに移動手段を95.13\%で判定でき,
% 移動手段補正モジュールを用いて移動手段を98.44\%で判定することが出来た.
% 本システムにより十分実用的な移動手段の判定を行える.
% また,教師データがないユーザを追加した場合では,
% 移動手段補正モジュールを用いずに移動手段を9.78\%で判定でき,
% 移動手段補正モジュールを用いて移動手段を10.28\%で判定することが出来た.
% 本システムでは,ユーザごとに教師データを作成し,移動手段の判定を行う必要があった.
% そして,教師データを作った端末と異なる端末に変更した場合では,
% 移動手段補正モジュールを用いずに移動手段を45.83\%で判定でき,
% 移動手段補正モジュールを用いて移動手段を47.28\%で判定することが出来た.
% 異なる端末で対応出来る移動手段と対応できない移動手段があった.
% 本システムでは,ユーザごとの教師データがある場合において高い精度で移動手段を判定することが出来たが,
% 教師データがないユーザでは精度が低かった.