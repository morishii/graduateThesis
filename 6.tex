% \chapter{予備実験}
% 本章では,加速度センサのみを用いた移動手段判定システムで使用する、
% 特徴量の選定と機械学習アルゴリズムの選定を目的として行った予備実験について述べる.

% \section{予備実験の概要}
% 本節では,特徴量の選定と機械学習アルゴリズムの選定を目的として行った予備実験の概要を述べる.
% はじめに,予備実験を行う目的について説明する.その後,予備実験の手順について説明する.

% \subsection{目的}
% 本システムでは第4章の設計で示したように,
% 特徴量の抽出手法として,時系列分割手法と周波数分割手法の2つの手法を選択した.
% また,機械学習アルゴリズムとしてBaysNet, J48, KStar, Multilayer Perceptron, RandomForest, SMO
% の6種類のアルゴリズムを選択した.
% これらの本システムに適用する特徴量と機械学習アルゴリズムの選定を目的とする.
% 本システムに適用し,最も分類精度が高い特徴量と機械学習アルゴリズムを,
% 本システムに採用する.

% \subsection{手順}
% 予備実験手順は以下の通りである.
% \begin{itembox}[l]{予備実験手順}
%  \begin{enumerate}
%   \item 加速度データの収集
%   \item 特徴量の抽出
%   \item 機械学習による分類
%  \end{enumerate}
% \end{itembox}

% \subsubsection*{加速度データの収集}
% まず,本システムに適用する加速度データを収集した.
% Galaxy Nexusを使用し,7種類の移動手段(歩行,走行,自転車,電車,バス,自動車,その他)を行っている際の加速度データを被験者5人に計測してもらいそれぞれ約300秒間記録した.
% 被験者は慶應義塾大学徳田研究室の5名とした.
% また,スマートフォンの所持位置はズボンのポケットの中,カバンの中,手の上の3種類で記録した.
% 7種類の移動手段を3種類の所持位置で5名の被験者に約300秒間記録したため,合計約9時間分の加速度データを収集した.
% \subsubsection*{特徴量の抽出}
% 次に,収集した加速度データから教師データを作成するために特徴量の抽出を行う.
% 第4章の設計で示した時系列分割手法と周波数分割手法に分け,特徴量を抽出した.
% 特徴量抽出のための時間幅(フレーム)を4秒間にして,
% 時系列分割手法を用いて特徴量を抽出した教師データと,
% 周波数分割手法を用いて特徴量を抽出した教師データの2種類を作成した.

% \subsubsection*{機械学習による分類}
% 最後に作成した教師データを用いて機械学習による分類を行う.
% 機械学習ソフトウェアのWekaに特徴量を抽出した教師データをセットし,
% 機械学習アルゴリズムとしてBaysNet, J48, KStar, Multilayer Perceptron, RandomForest, SMO
% の6種類のアルゴリズムを切り替えて分類精度を評価した.
% なお,分類精度の評価としては交差検証を用いた評価を行った.

% \section{予備実験結果}
% 予備実験結果を表\ref{preeval1}, 図\ref{preeval11}に示す.
% 各アルゴリズムの特徴量ごとの混同行列をそれぞれ
% 表\ref{matrix11}-\ref{matrix62}
% に示す.
% 混同行列は,行が実際の移動手段を表し,列が本システムが分類した移動手段を表している.

% % また,BayesNetの時系列分割手法の混同行列を図\ref{matrix11}に,
% % BayesNetの周波数分割手法の混同行列を図\ref{matrix12}に,
% % J48の時系列分割手法の混同行列を図\ref{matrix21}に,
% % J48の周波数分割手法の混同行列を図\ref{matrix22}に,
% % KStarの時系列分割手法の混同行列を図\ref{matrix31}に,
% % KStarの周波数分割手法の混同行列を図\ref{matrix32}に,
% % Multilayer Perceptronの時系列分割手法の混同行列を図\ref{matrix41}に,
% % Multilayer Perceptronの周波数分割手法の混同行列を図\ref{matrix42}に,
% % RandomForestの時系列分割手法の混同行列を図\ref{matrix51}に,
% % RandomForestの周波数分割手法の混同行列を図\ref{matrix52}に,
% % SMOの時系列分割手法の混同行列を図\ref{matrix61}に,
% % SMOの周波数分割手法の混同行列を図\ref{matrix62}に,


% 最も分類精度が高いのは特徴量の抽出手法に時系列分割手法を,
% 機械学習のアルゴリズムにRandomForestを用いた場合で99.04\%であった.
% 逆に,最も分類精度が低かったのは特徴量の抽出手法に周波数分割手法を,
% 機械学習のアルゴリズムにBayesNetを用いた場合で62.87\%であった.

% また,特徴量の抽出手法を時系列分割手法と周波数分割手法で比較すると,
% 時系列分割手法のほうが,SMO以外の機械学習アルゴリズムで分類精度が高かった.
% さらに,混同行列からは,どの特徴量の抽出手法,どの機械学習アルゴリズムでも歩行,走行,自転車の
% 分類精度は高かったが,電車,バス,自動車,その他は比較的分類精度が低かった.


% \begin{table}[htbp]
%  \begin{center}
%  \caption{予備実験結果}
%  \label{preeval1}
%   \begin{tabular}{l|rrr}
%   \hline
%   & \multicolumn{2}{c}{分類精度}\\
%   アルゴリズム & 時系列分割手法 & 周波数分割手法 \\
%   \hline \hline
%   BayesNet              & 95.13\% & 62.87\% \\
%   J48                   & 98.04\% & 82.87\% \\
%   KStar                 & 98.25\% & 82.73\% \\
%   Multilayer Perceptron & 93.85\% & 77.96\% \\
%   Random Forest         & 99.04\% & 85.25\% \\
%   SMO                   & 70.75\% & 85.84\% \\
%   \hline
%   \end{tabular}
%  \end{center}
% \end{table}

% \begin{figure}[htbp]
%   \begin{center}
%    \includegraphics[width=120mm]{graph/6/eval1.pdf}
%    \caption{予備実験結果}
%    \label{preeval11}
%   \end{center}
% \end{figure}

% \begin{table}[htbp]
%  \begin{center}
%  \caption{BayesNetの混同行列(時系列分割手法)}
%  \label{matrix11}
%   \begin{tabular}{c|rrrrrrr}
%   \hline
%    & 歩行 & 走行 & 自転車 & 電車 & バス & 自動車 & その他 \\
%   \hline \hline
%   歩行   & 99.45\% & 0.45\%  & 0.00\%  & 0.00\%  & 0.00\%  & 0.00\%  & 0.00\% \\
%   走行   & 1.27\%  & 98.73\% & 0.00\%  & 0.00\%  & 0.00\%  & 0.00\%  & 0.00\% \\
%   自転車 & 0.50\%  & 0.00\%  & 96.17\% & 0.67\%  & 1.00\%  & 1.67\%  & 0.00\% \\
%   電車   & 0.50\%  & 0.13\%  & 0.38\%  & 92.88\% & 2.88\%  & 2.50\%  & 0.75\% \\
%   バス   & 0.60\%  & 0.00\%  & 1.80\%  & 6.20\%  & 83.00\% & 6.60\%  & 1.80\% \\
%   自動車 & 0.00\%  & 0.00\%  & 1.17\%  & 0.83\%  & 3.00\%  & 95.00\% & 0.00\% \\
%   その他 & 0.00\%  & 0.00\%  & 0.00\%  & 1.50\%  & 5.13\%  & 0.00\%  & 93.28\% \\
%   \hline
%   \end{tabular}
%  \end{center}
% \end{table}

% \begin{table}[htbp]
%  \begin{center}
%  \caption{BayesNetの混同行列(周波数分割手法)}
%  \label{matrix12}
%   \begin{tabular}{c|rrrrrrr}
%   \hline
%    & 歩行 & 走行 & 自転車 & 電車 & バス & 自動車 & その他 \\
%   \hline \hline
%   歩行   & 47.91\% & 13.82\% & 22.55\% & 0.00\%  & 0.00\%  & 15.73\% & 0.00\% \\
%   走行   & 25.36\% & 74.09\% & 0.55\%  & 0.00\%  & 0.00\%  & 0.00\%  & 0.00\% \\
%   自転車 & 8.01\%  & 8.51\%  & 66.28\% & 0.50\%  & 0.00\%  & 16.53\% & 0.17\% \\
%   電車   & 0.50\%  & 0.13\%  & 0.63\%  & 66.38\% & 2.13\%  & 3.63\%  & 26.63\% \\
%   バス   & 0.20\%  & 0.00\%  & 6.60\%  & 11.80\% & 33.00\% & 15.20\% & 33.20\% \\
%   自動車 & 0.00\%  & 0.00\%  & 9.17\%  & 13.67\% & 0.50\%  & 62.00\% & 14.67\% \\
%   その他 & 0.00\%  & 0.00\%  & 0.13\%  & 14.75\% & 0.13\%  & 3.63\%  & 81.38\% \\
%   \hline
%   \end{tabular}
%  \end{center}
% \end{table}

% \begin{table}[htbp]
%  \begin{center}
%  \caption{J48の混同行列(時系列分割手法)}
%  \label{matrix21}
%   \begin{tabular}{c|rrrrrrr}
%   \hline
%    & 歩行 & 走行 & 自転車 & 電車 & バス & 自動車 & その他 \\
%   \hline \hline
%   歩行   & 99.45\% & 0.09\%  & 0.27\%  & 0.09\%  & 0.09\%  & 0.00\%  & 0.00\% \\
%   走行   & 0.09\%  & 99.82\% &  0.00\% & 0.09\%  & 0.00\%  & 0.00\%  & 0.00\% \\
%   自転車 & 0.67\%  & 0.00\%  & 98.00\% &  0.33\% & 0.50\%  & 0.50\%  & 0.00\% \\
%   電車   & 0.25\%  & 0.13\%  & 0.38\%  & 96.00\% & 1.88\%  & 0.38\%  & 1.00\% \\
%   バス   & 0.00\%  & 0.00\%  & 1.60\%  & 2.60\%  & 93.80\% & 1.80\%  & 0.20\% \\
%   自動車 & 0.00\%  & 0.00\%  & 1.00\%  & 0.33\%  & 1.17\%  & 96.83\% & 0.67\% \\
%   その他 & 0.00\%  & 0.00\%  & 0.00\%  & 0.63\%  & 0.13\%  & 0.00\%  & 99.25\% \\

%   \hline
%   \end{tabular}
%  \end{center}
% \end{table}

% \begin{table}[htbp]
%  \begin{center}
%  \caption{J48の混同行列(周波数分割手法)}
%  \label{matrix22}
%   \begin{tabular}{c|rrrrrrr}
%   \hline
%    & 歩行 & 走行 & 自転車 & 電車 & バス & 自動車 & その他 \\
%   \hline \hline
%   歩行   & 98.55\% & 1.00\%  & 0.45\%  & 0.00\%  & 0.00\%  & 0.00\%  & 0.00\% \\
%   走行   & 1.18\%  & 98.64\% & 0.18\%  & 0.00\%  & 0.00\%  & 0.00\%  & 0.00\% \\
%   自転車 & 0.67\%  & 0.50\%  & 89.33\% & 1.83\%  & 0.33\%  & 6.00\%  & 1.33\% \\
%   電車   & 0.00\%  & 0.00\%  & 1.38\%  & 70.13\% & 5.13\%  & 6.00\%  & 17.38\% \\
%   バス   & 0.40\%  & 0.20\%  & 1.00\%  & 8.40\%  & 72.40\% & 12.00\% & 5.60\% \\
%   自動車 & 0.00\%  & 0.00\%  & 9.00\%  & 8.00\%  & 11.00\% & 62.00\% & 10.00\% \\
%   その他 & 0.00\%  & 0.00\%  & 0.75\%  & 15.88\% & 2.50\%  & 11.13\% & 69.75\% \\
%   \hline
%   \end{tabular}
%  \end{center}
% \end{table}

% \begin{table}[htbp]
%  \begin{center}
%  \caption{KStarの混同行列(時系列分割手法)}
%  \label{matrix31}
%   \begin{tabular}{c|rrrrrrr}
%   \hline
%    & 歩行 & 走行 & 自転車 & 電車 & バス & 自動車 & その他 \\
%   \hline \hline
%   歩行   & 99.55\% & 0.00\%  & 0.00\%  & 0.18\%  & 0.00\%  & 0.00\%  & 0.27\% \\
%   走行   & 0.18\%  & 99.36\% & 0.00\%  & 0.00\%  & 0.36\%  & 0.00\%  & 0.09\% \\
%   自転車 & 0.00\%  & 0.00\%  & 98.00\% & 0.67\%  & 0.17\%  & 0.67\%  & 0.50\% \\
%   電車   & 0.13\%  & 0.00\%  & 0.13\%  & 96.88\% & 1.75\%  & 0.50\%  & 0.63\% \\
%   バス   & 0.00\%  & 0.00\%  & 0.20\%  & 4.20\%  & 94.60\% & 0.80\%  & 0.20\% \\
%   自動車 & 0.00\%  & 0.00\%  & 0.00\%  & 2.50\%  & 0.83\%  & 96.67\% & 0.00\% \\
%   その他 & 0.00\%  & 0.00\%  & 0.00\%  & 0.00\%  & 0.00\%  & 0.00\%  & 100.00\% \\
%   \hline
%   \end{tabular}
%  \end{center}
% \end{table}


% \begin{table}[htbp]
%  \begin{center}
%  \caption{KStarの混同行列(周波数分割手法)}
%  \label{matrix32}
%   \begin{tabular}{c|rrrrrrr}
%   \hline
%    & 歩行 & 走行 & 自転車 & 電車 & バス & 自動車 & その他 \\
%   \hline \hline
% 歩行 & 92.09\% & 0.00\% & 3.73\% & 0.91\% & 0.45\% & 2.73\% & 0.09\% \\
% 走行 & 0.82\% & 96.45\% & 2.64\% & 0.00\% & 0.00\% & 0.09\% & 0.00\% \\
% 自転車 & 0.00\% & 0.00\% & 79.17\% & 1.33\% & 1.33\% & 15.50\% & 2.67\% \\
% 電車 & 0.00\% & 0.00\% & 0.25\% & 71.50\% & 3.13\% & 2.13\% & 23.00\% \\
% バス & 0.00\% & 1.53\% & 0.17\% & 13.10\% & 48.81\% & 3.91\% & 32.48\% \\
% 自動車 & 0.00\% & 0.00\% & 0.50\% & 5.17\% & 6.00\% & 68.33\% & 20.00\% \\
% その他 & 0.00\% & 0.00\% & 0.00\% & 3.50\% & 2.50\% & 2.50\% & 91.50\% \\

%   \hline
%   \end{tabular}
%  \end{center}
% \end{table}

% \begin{table}[htbp]
%  \begin{center}
%  \caption{Multilayer Perceptronの混同行列(時系列分割手法)}
%  \label{matrix41}
%   \begin{tabular}{c|rrrrrrr}
%   \hline
%    & 歩行 & 走行 & 自転車 & 電車 & バス & 自動車 & その他 \\
%   \hline \hline
%   歩行 & 99.73\% & 0.09\% & 0.00\% & 0.09\% & 0.09\% & 0.00\% & 0.00\% \\
%   走行 & 0.45\% & 99.45\% & 0.00\% & 0.00\% & 0.00\% & 0.09\% & 0.00\% \\
%   自転車 & 0.33\% & 0.00\% & 96.83\% & 1.00\% & 0.50\% & 1.33\% & 0.00\% \\
%   電車 & 0.25\% & 1.25\% & 0.13\% & 93.50\% & 2.00\% & 0.50\% & 2.38\% \\
%   バス  & 2.80\% & 2.60\% & 1.80\% & 15.20\% & 71.00\% & 1.60\% & 5.00\% \\
%   自動車 & 0.17\% & 0.00\% & 1.17\% & 0.83\% & 1.33\% & 96.33\% & 0.17\% \\
%   その他 & 0.75\% & 3.00\% & 0.63\% & 2.63\% & 0.00\% & 4.38\% & 88.63\% \\
%   \hline
%   \end{tabular}
%  \end{center}
% \end{table}

% \begin{table}[htbp]
%  \begin{center}
%  \caption{Multilayer Perceptronの混同行列(周波数分割手法)}
%  \label{matrix42}
%   \begin{tabular}{c|rrrrrrr}
%   \hline
%    & 歩行 & 走行 & 自転車 & 電車 & バス & 自動車 & その他 \\
%   \hline \hline
%   歩行 & 99.45\% &  0.00\% & 0.55\% & 0.00\% & 0.00\% & 0.00\% & 0.00\% \\
%   走行 & 0.36\% & 99.64\% &  0.00\% & 0.00\% & 0.00\% & 0.00\% & 0.00\% \\
%   自転車 & 0.00\% & 0.17\% & 94.65\% &  1.00\% & 0.33\% & 3.34\% & 0.50\% \\
%   電車 & 0.00\% & 0.00\% & 0.75\% & 80.13\% &  1.50\% & 2.50\% & 15.13\% \\
%   バス & 0.00\% & 0.00\% & 0.40\% & 33.80\% &  46.80\% &  1.80\% & 17.20\% \\
%   自動車 & 0.00\% & 0.00\% & 5.50\% & 23.33\% &  5.67\% & 46.17\% &  19.33\% \\
%   その他 & 0.00\% & 0.00\% & 0.13\% & 48.25\% &  0.00\% & 4.38\% & 47.25\% \\
%   \hline
%   \end{tabular}
%  \end{center}
% \end{table}




% \begin{table}[htbp]
%  \begin{center}
%  \caption{Random Forestの混同行列(時系列分割手法)}
%  \label{matrix51}
%   \begin{tabular}{c|rrrrrrr}
%   \hline
%   & 歩行 & 走行 & 自転車 & 電車 & バス & 自動車 & その他 \\
%   \hline \hline
%   歩行 & 99.73\% & 0.18\% & 0.09\% & 0.00\% & 0.00\% & 0.00\% & 0.00\% \\
%   走行 & 0.09\% & 99.91\% & 0.00\% & 0.00\% & 0.00\% & 0.00\% & 0.00\% \\
%   自転車 & 0.17\% & 0.00\% & 98.50\% & 0.67\% & 0.17\% & 0.00\% & 0.50\% \\
%   電車 & 0.00\% & 0.13\% & 0.25\% & 98.13\% & 1.25\% & 0.25\% & 0.00\% \\
%   バス & 0.20\% & 0.00\% & 0.80\% & 2.00\% & 96.80\% & 0.00\% & 0.20\% \\
%   自動車 & 0.00\% & 0.00\% & 0.33\% & 0.33\% & 0.67\% & 98.67\% & 0.00\% \\
%   その他 & 0.00\% & 0.00\% & 0.00\% & 0.13\% & 0.00\% & 0.00\% & 99.88\% \\
%   \hline
%   \end{tabular}
%  \end{center}
% \end{table}

% \begin{table}[htbp]
%  \begin{center}
%  \caption{Random Forestの混同行列(周波数分割手法)}
%  \label{matrix52}
%   \begin{tabular}{c|rrrrrrr}
%   \hline
%    & 歩行 & 走行 & 自転車 & 電車 & バス & 自動車 & その他 \\
%   \hline \hline
%   歩行 & 97.82\% & 1.45\% & 0.64\% & 0.00\% & 0.00\% & 0.09\% & 0.00\% \\
%   走行 & 0.36\% & 99.64\% & 0.00\% & 0.00\% & 0.00\% & 0.00\% & 0.00\% \\
%   自転車 & 4.00\% & 1.17\% & 91.50\% & 0.33\% & 0.50\% & 2.00\% & 0.50\% \\
%   電車 & 0.38\% & 0.00\% & 0.88\% & 72.25\% & 3.25\% & 3.38\% & 19.88\% \\
%   バス & 0.00\% & 0.00\% & 2.00\% & 9.60\% & 69.00\% & 7.80\% & 11.60\% \\
%   自動車 & 0.50\% & 0.00\% & 6.17\% & 8.67\% & 5.83\% & 64.33\% & 14.50\% \\
%   その他 & 0.00\% & 0.00\% & 0.13\% & 12.00\% & 1.25\% & 4.25\% & 82.38\% \\
%   \hline
%   \end{tabular}
%  \end{center}
% \end{table}

% \begin{table}[htbp]
%  \begin{center}
%  \caption{SMOの混同行列(時系列分割手法)}
%  \label{matrix61}
%   \begin{tabular}{c|rrrrrrr}
%   \hline
%    & 歩行 & 走行 & 自転車 & 電車 & バス & 自動車 & その他 \\
%   \hline \hline
%   歩行 & 95.64\% & 0.00\% & 1.82\% & 0.00\% & 0.00\% & 0.00\% & 2.55\% \\
%   走行 & 0.55\% & 99.45\% & 0.00\% & 0.00\% & 0.00\% & 0.00\% & 0.00\% \\
%   自転車 & 0.33\% & 0.00\% & 69.50\% & 9.67\% & 16.67\% & 2.33\% & 1.50\% \\
%   電車 & 0.25\% & 0.13\% & 9.63\% & 42.13\% & 0.00\% & 12.50\% & 35.38\% \\
%   バス & 3.80\% & 0.00\% & 2.20\% & 20.60\% & 38.00\% & 21.00\% & 14.40\% \\
%   自動車 & 0.00\% & 0.00\% & 0.50\% & 29.17\% & 16.67\% & 53.00\% & 0.67\% \\
%   その他 & 0.00\% & 0.00\% & 0.00\% & 14.63\% & 11.38\% & 13.63\% & 60.38\% \\
%   \hline
%   \end{tabular}
%  \end{center}
% \end{table}

% \begin{table}[htbp]
%  \begin{center}
%  \caption{SMOの混同行列(周波数分割手法)}
%  \label{matrix62}
%   \begin{tabular}{c|rrrrrrr}
%   \hline
%    & 歩行 & 走行 & 自転車 & 電車 & バス & 自動車 & その他 \\
%   \hline \hline
%   歩行 & 100.00\% & 0.00\% & 0.00\% & 0.00\% & 0.00\% & 0.00\% & 0.00\% \\
%   走行 & 0.00\% & 100.00\% & 0.00\% & 0.00\% & 0.00\% & 0.00\% & 0.00\% \\
%   自転車 & 0.00\% & 0.00\% & 97.17\% & 0.33\% & 0.00\% & 1.83\% & 0.67\% \\
%   電車 & 0.13\% & 0.00\% & 0.38\% & 64.75\% & 0.38\% & 1.50\% & 32.88\% \\
%   バス & 0.00\% & 0.00\% & 0.00\% & 12.80\% & 50.80\% & 2.40\% & 34.00\% \\
%   自動車 & 0.00\% & 0.00\% & 1.17\% & 6.67\% & 0.50\% & 71.17\% & 20.50\% \\
%   その他 & 0.00\% & 0.00\% & 0.00\% & 5.63\% & 0.00\% & 2.00\% & 92.38\% \\
%   \hline
%   \end{tabular}
%  \end{center}
% \end{table}


% \clearpage

% \section{考察}
% \subsubsection*{予備実験での分類精度}
% 本研究が提案する加速度センサのみを用いた移動手段判定システムにおいて,
% 最も分類精度が高かったのは特徴量の抽出手法に時系列分割手法を用いて,
% 機械学習アルゴリズムにRandom Forestを用いた場合だった.
% Random Forestは多数の決定木を生成し,それら複数の決定木から多数決によって分類するアルゴリズムである.
% 決定木を弱学習機としているため,Random Forestはノイズに対して強いという特徴がある.

% \subsubsection*{分類が容易な移動手段}
% 混同行列からは歩行,走行,自転車の場合は分類精度が高いが,
% それに比べ,電車,バス,自動車,その他の場合は分類精度が低かった.
% これは,歩行や走行,自転車の時の加速度の変化量が大きいため,抽出した特徴量に大きく違いが見られたため
% 分類精度が高かったと考えられる.
% 逆に,電車やバス,自動車,その他の場合は加速度の変化量が小さいために,分類精度が低くなってしまったと考えられる.
% 加速度センサのみを用いて移動手段を判定する際には,
% 加速度の変化量が大きいと分類精度が高くなり,
% 加速度の変化量が小さいと分類精度が低くなると言える.

% \subsubsection*{特徴量の抽出手法}
% 今回抽出した特徴量は,
% 時系列分割手法では14個のパラメータを生成し,
% 周波数分割手法では768個のパラメータを生成している.
% 機械学習に適用するパラメータ数が多ければ多いほど,機械学習の分類における計算量が多くなる.
% その結果,分類する際に時間がかかってしまうという問題や,デバイスの消費電力が増加するという問題がある.
% スマートフォンなど計算能力が限られたデバイスでは,機械学習で使用する特徴量のパラメータ数は重要である.
% 今回の結果では時系列分割手法のほうが分類精度が高く,パラメータ数も少ないために,
% 移動手段を判定する際には周波数分割手法よりも時系列分割手法が優れていると考えられる.

% \subsubsection*{機械学習アルゴリズムの計算量}
% 機械学習のアルゴリズムごとにも計算量の違いがある.
% BayesNetやJ48, Random Forest, SMOでは,計算量が少なく瞬時に分類を行うことができるが,
% KStarやMultilayer Perceptronでは計算量が多く分類に多くの時間を要する.
% 分類に時間がかかってしまうと,スマートフォン上でのリアルタイムな移動手段の判定を行うことが出来ない.
% 今回の予備実験では,特徴量抽出の時間幅(フレーム)を4秒間に設定したため,4秒毎に移動手段の判定を行っている.
% 移動手段の判定をする際に,機械学習の分類に4秒以上かかってしまうと判定が追いつかなくなってしまい,結果としてリアルタイムな移動手段の判定が不可能になる.
% また,計算量が多いアルゴリズムを使用すると,デバイスの消費電力が増加してしまうという問題もある.
% 今回最も分類精度が高かったRandom Forestは,計算量が少ないアルゴリズムのため
% スマートフォン上での移動手段の判定に適していると言える.

% よって本システムでは特徴量の抽出手法に時系列分割手法を用いて,
% 機械学習アルゴリズムにRandom Forestを用いる.


% \section{まとめ}
% 本章では,加速度センサのみを用いた移動手段判定システムに使用する
% 特徴量の選定と機械学習アルゴリズムの選定を目的とした予備実験を行った.
% 予備実験の結果,最も分類精度が高かったのは特徴量に時系列分割手法を用いて,
% 機械学習アルゴリズムにRandom Forestを用いた場合で,99.04\%だった.
% 本システムでは,特徴量に時系列分割手法を用いて,機械学習アルゴリズムにRandom Forestを用いる.

% 次章では,評価用データを用いた評価実験について述べる.